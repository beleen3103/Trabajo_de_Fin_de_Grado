\section{Classification of forecasting}

There are multiple ways of classification for the different types of forecasting methods used nowadays. Time intervals, applied models,accuracy wanted for the output or number of units monitored are some examples of the many forms to classify forecasting. This, added to the fact that there are new methods of forecasting everyday, makes their classification a quite laborious task.

Nonetheless, in (https://sci-hub.mksa.top/10.1109/PMAPS.2016.7764085) is exposed a wide classification of most of the current wind forecasting methods. Therefore, we will use this classification as a guide, and also try to complement it with other relevant papers.

\subsection{Classified by time scale}

Forecasting tries to guess the wind's behaviour within a determinate time interval, from scarce seconds to months, or even years. Keeping that in mind, depending on the time interval used, we will choose the best forecasting method that matches our interests.

Nowadays, there seems to be a consensus about the \textbf{division in ranges} of time, used in almost every paper:

\begin{enumerate}
    \item Immediate-short-term
    \item Short-term
    \item Medium-term
    \item Long-term
\end{enumerate}

However, is when we try to unify the amount of time that defines each interval that we start to have some difficult problems. As it is said in (https://sci-hub.mksa.top/https://doi.org/10.1002/for.2657), a lot of authors from the last decade have shared their particular definition, i.e, (https://sci-hub.mksa.top/10.1109/naps.2010.5619586) or (https://www.sciencedirect.com/science/article/pii/S0959652619333177\#bib49).

In this paper we will take a classification similar to the proposed in (https://sci-hub.mksa.top/https://doi.org/10.1002/for.2657) and (https://sci-hub.mksa.top/10.1109/IEECON.2018.8712262), as follows:

\begin{enumerate}
    \item \textbf{Immediate-short-term:}  [a few seconds, one hour)
    \item \textbf{Short-term:} [one hour, up to six hours)
    \item \textbf{Medium-term:} [up to six hours, a couple of days)
    \item \textbf{Long-term:} [a couple of days onward]
\end{enumerate}

As mentioned before, the 