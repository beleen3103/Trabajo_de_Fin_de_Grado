\section{Classification of forecasting}

There are multiple ways of classification for the different types of forecasting methods used nowadays. Time intervals, applied models,accuracy wanted for the output or number of units monitored are some examples of the many forms to classify forecasting. This, added to the fact that there are new methods of forecasting everyday, makes their classification a quite laborious task.

Nonetheless, in (https://sci-hub.mksa.top/10.1109/PMAPS.2016.7764085) is exposed a wide classification of most of the current wind forecasting methods. Therefore, we will use this classification as a guide, and also try to complement it with other relevant papers.

\subsection{Classified by time scale}

Forecasting tries to guess the wind's behaviour within a determinate time interval, from scarce seconds to months, or even years. Keeping that in mind, depending on the time interval used, we will choose the best forecasting method that matches our interests.

Nowadays, there seems to be a consensus about the \textbf{division in ranges} of time, used in almost every paper:

\begin{enumerate}
    \item Immediate-short-term
    \item Short-term
    \item Medium-term
    \item Long-term
\end{enumerate}

However, is when we try to unify the amount of time that defines each interval that we start to have some difficult problems. As it is said in (https://sci-hub.mksa.top/https://doi.org/10.1002/for.2657), a lot of authors from the last decade have shared their particular definition, i.e, (https://sci-hub.mksa.top/10.1109/naps.2010.5619586) or (https://www.sciencedirect.com/science/article/pii/S0959652619333177\#bib49).

In this paper we will take a classification similar to the proposed in (https://sci-hub.mksa.top/https://doi.org/10.1002/for.2657) and (https://sci-hub.mksa.top/10.1109/IEECON.2018.8712262), as follows:

\begin{enumerate}
    \item \textbf{Immediate-short-term:}  [a few seconds, one hour)
    \item \textbf{Short-term:} [one hour, up to six hours)
    \item \textbf{Medium-term:} [up to six hours, a couple of days)
    \item \textbf{Long-term:} [a couple of days onward]
\end{enumerate}

Moreover, in (https://sci-hub.mksa.top/10.1109/IEECON.2018.8712262) is mentioned the common uses of each interval of time too:

\begin{enumerate}
    \item \textbf{Immediate-short-term:}  Regulation Actions and Electricity market clearing
    \item \textbf{Short-term:} Load increment/decrement decisions and Economic Load Dispatch Planning
    \item \textbf{Medium-term:} Generator offline/online decisions and Operational Security in day-ahead electricity market
    \item \textbf{Long-term:} Maintenance scheduling to obtain optimum operating cost, Reserve requirement decisions and Unit Commitment decisions
\end{enumerate}

\subsection{Classified by prediction model}

Once we have picked a range of time to work on, there are plenty of methods to choose for our forecasting, varying in the nature of their prediction technique: replicating the natural conditions of weather given some physical parameters, the relation between them or a hybrid of the techniques mentioned before. For this matter, we will be using the classification shown in (https://sci-hub.mksa.top/10.1016/j.jclepro.2020.124628):

\subsubsection{Physical methods}
Physical approaches are mainly based on NWP models, which utilize NWP data and geographic data to calculate wind speed at the height of wind turbine. However, they are greatly sensitive to initial conditions, which might result in deviations of wind speed values. Hence, they are often synthesized with other models to enhance forecasting precision (https://ieeexplore.ieee.org/document/9146490).

\subsubsection{Statistical methods}
Statistical approaches aim at establishing accurate mapping between inputs (i.e., NWP data, historical data, geographic data, etc.) and outputs (wind speed or power), which mainly consist of time series analysis approaches, Kalman Filtering and machine learning approaches.
Time series analysis approaches can predict wind speed in next few minutes based on historical data without an accurate system model. Some of the statistical methods based on time series analysis are: persistence, grey prediction, ARMA and ARIMA methods. 

Persistence is the simple method of forecasting methods, using the recent wind speed or power point observations to set the next point prediction value. The method is suitable for short term, but the prediction accuracy is poor and is used now as a benchmark to evaluate the accuracy of other methods.

Grey system theory is a method to solve uncertain problems by studying a small amount of information and data in order to use a grey model for sample modelling with a small amount of data. It's applied to forecast immediate short term wind speed on fast tracking real-time wind speed conditions, but it has a poor effect on mutation point prediction.

Time series model (ARMA, ARIMA) requires a lot of high-quality historical data to model, after model identification, parameter estimation and model checking, to determine a mathematical model to describe the studied time series and finally deduces the prediction model of wind power. This method is mainly used or immediate short term forecasting because if the prediction scale becomes too large, the accuracy will decline. ARIMA method is an extension of ARMA which requires initial historical data if initial data are not smooth.

Kalman Filtering method is a statistically optimal sequential estimation procedure for dynamics systems, which relies on a series of mathematical calculations. Wind observations are recursively combined with recent forecast using weights to minimize corresponding biases.

Machine learning approaches can build inductive models between inputs and outputs based on various learning rules without detailed system information. Some of the statistical methods using machine learning are: neural networks, support vector machines (SVM), wavelet analysis and fuzzy logic.

Neural networks are massively parallel distributed processing systems as an analogy of the human brain information processing mechanism developed from the basis of modern biology research. The advantages of using neural networks are: high execution speed, strong robustness, the ability to learn, fast calculation and, for nonlinear modelling of complex systems, they represent a more practical an economic way of forecasting. However, neural networks require a long time to train and high similarity of training samples, and they can easily fall into local optimums.

Support Vector Machines (SVM) are a machine learning method based on statistical theory, rigorous mathematical theory, to achieve the structural risk minimisation, taking the empirical risk minimisation and fiducial range into account. The model has a strong promotional, being able to solve nonlinear regression estimation problem.

Wavelet analysis is based on the theory of harmonic analysis, functional analysis and Fourier analysis, which implements the frequency analysis of the signals by the telescopic pan operation. The method subdivides the frequency of signal at low frequency, subdivides time of signal at high frequency and realises the good localisation and multi-resolution characteristics in the frequency domain and time-domain, using signal analysis often called "mathematics microscope".

Fuzzy logic method can be used to predict the radiation, which uses the value between the interval [0,1] and long, medium and short fuzzy variables to explain the relationship between values, taking advantage of fuzzy logic and expertise of the forecasting staff to for  to form a fuzzy rule base with data and language, and then choose the linear model to approximate nonlinear dynamic changes op wind speed.

\subsubsection{Combined Methods}

For combined methods, the basic idea is to combine forecasting methods with different weights, to make full use of the information provided by all models and integrated process data to finally obtain combined forecasting results. The most universal methods are: equally weighted average method, the optimal weight coefficient method, Gaussian process regression, Support Vector Regression and Bayesian method.

\subsection{Classified by the accuracy of output data}

Wind speed and power forecasting are also known as deterministic/point prediction, which are inevitably influenced by uncertain factors, such as wind speed randomness, observation precision and the status of wind turbines. Thus, uncertainty forecasting has drawn ever-increasing attentions to decrease decision risk to power system. The principal methods used in uncertainty forecasting are: probabilistic forecasting, risk index forecasting and space-time scenario forecasting.

\subsubsection{Probabilistic methods}



\subsection{Classified by the prediction physical quantity}

Depending on the wind parameter that we are trying to forecast, there are two methods to obtain the wind power prediction, although they are directly related. The first one is to forecast the wind speed, and then calculate the wind power predicted. The second one, more obvious, is to forecast the wind power directly.

\subsection{Classified by the input data}

\subsection{Classified by the predicted range}