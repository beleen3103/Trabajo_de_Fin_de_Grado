\section{Introduction}

In recent years, the importance of clean and renewable energy has been increasing due to pollution increase and environmental degradation. In that context wind, solar, wave or biomass energy stands out. Specifically, wind energy has been considered as one of the most promising alternatives.

Nevertheless, we’re talking about energy in a certain way unstable due to continuous variation of energy’s speed or  temperature among others. The uncertainty generated by this energy’s production clearly affects its stability and increases the cost of its productions. It’s because of this that accurate forecasting  positively affects wind energy’s development and its commerce in certain markets (https://sci-hub.mksa.top/10.1016/j.ijepes.2015.07.039).

Forecasting has been an essential part of the energetic industry’s development since a long time ago. In the case of wind energy, it’s characterized by randomness and wind’s intermittence. That’s why that business depends on wind forecasting for price calculation, to ensure a stable and constant use or to make future plans.

Within the European framework, some ambitious objectives have been set to promote the use of renewable energies. European Union (EU) has gone beyond the Directive requirements

ashfkhdsbhsvoavpavpavbis